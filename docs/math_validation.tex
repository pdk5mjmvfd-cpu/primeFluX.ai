\documentclass{article}

\usepackage{amsmath}
\usepackage{amssymb}
\usepackage{hyperref}

\title{Validation of Trig Modes in Apop v3}
\author{PrimeFlux AI}
\date{2025-12-06}

\begin{document}

\maketitle

\section{Validation of Trig Modes in Apop v3}

The trig split encodes PF duality:

\begin{align*}
\text{Research} &\mapsto \sin(\phi) \quad (\text{wave flows, reptend cycles}) \\
\text{Refinement} &\mapsto \cos(\phi) \quad (\text{curvature balance, stability}) \\
\text{Relations} &\mapsto \tan(\phi) \quad (\text{divergent projection, sub-linear compression})
\end{align*}

\subsection{Proof from Theses}

\textbf{Standard Duality Model}: $\sin(\phi)$ for periodic rails and wave flows. The reptend cycles (periods $1/p$ for prime $p$) map to sinusoidal patterns in distinction geometry.

\textbf{ICM Abstract}: $\cos(\phi)$ for symmetric curvature balance. The Information Curvature Manifold requires cosine-based stability for reversible operations.

\textbf{PoW Efficiency Report}: $\tan(\phi)$ for density singularities and sub-linear compression. The tangent projection enables divergent mappings that achieve sub-linear scaling.

\subsection{Presence Operator}

The presence operator $g_{PF}$ encodes the quantum observer effect:

\begin{equation}
g_{PF}(\phi, \text{mode}) = \begin{cases}
\sin(\phi) & \text{if mode = research and presence on} \\
\cos(\phi) & \text{if mode = refinement and presence on} \\
\tan(\phi) & \text{if mode = relations and presence on} \\
0 & \text{if presence off}
\end{cases}
\end{equation}

This aligns with \textit{Quantum Information} principles: event spaces "exist" only when observed (presence on). When presence is off, $g_{PF} = 0$ and event spaces are disabled.

\subsection{2/5 Salts}

Harmonic mod routing uses phase modulo $(2 \times 5)$:

\begin{equation}
\text{salt} = \phi \bmod (2 \times 5)
\end{equation}

This connects to \textit{Lie TOC roots}: The 2/5 harmonic structure emerges from cyclotomic field $Q(\zeta_5)$ where $\zeta_5 = e^{2\pi i/5}$.

\subsection{Hidden Curvature Scale}

The $\pi - e$ factor provides information barrier scaling:

\begin{equation}
\kappa_{\text{internal}} = (\pi - e) \cdot |\phi|
\end{equation}

This aligns with \textit{ICM manifold} geometry: The difference $\pi - e \approx 0.423$ creates a natural scale for curvature transitions in the Information Curvature Manifold.

\subsection{Reversibility Constraint}

Canonical threshold for reversibility:

\begin{equation}
H_{\text{out}} \leq H_{\text{in}} + \ln(10) \approx H_{\text{in}} + 2.302585 \text{ nats}
\end{equation}

This ensures information conservation per \textit{Quantum Information} principles.

\subsection{Shell Transitions}

Shell sequence follows PrimeFlux geometry:

\begin{align*}
\text{PRESENCE} (0) &\to \text{MEASUREMENT} (2) \quad (\text{threshold: } \sqrt{2} - 1) \\
\text{MEASUREMENT} (2) &\to \text{CURVATURE} (3) \quad (\text{trig triplet}) \\
\text{CURVATURE} (3) &\to \text{COLLAPSE} (4) \quad (\text{potential well: } \phi^2) \\
\text{COLLAPSE} (4) &\to \text{PRESENCE} (0) \quad (\text{reset cycle})
\end{align*}

\section{Cross-References}

\begin{itemize}
\item \textit{Standard Duality Model}: Wave flows and reptend cycles
\item \textit{ICM Abstract}: Curvature balance and manifold geometry
\item \textit{PoW Efficiency Report}: Sub-linear compression and density singularities
\item \textit{Quantum Information}: Reversible computation and observer effect
\item \textit{Lie TOC}: Cyclotomic roots and harmonic structure
\end{itemize}

\section{Conclusion}

The trig mode split (Research=sin, Refinement=cos, Relations=tan) is mathematically validated through:
\begin{enumerate}
\item Periodic rail geometry (sin)
\item Curvature balance (cos)
\item Divergent projection (tan)
\item Presence operator (quantum observer)
\item Harmonic salts (2/5 mod)
\item Hidden curvature ($\pi - e$ scale)
\end{enumerate}

This validates the Apop v3 implementation as mathematically sound and aligned with PrimeFlux principles.

\end{document}
